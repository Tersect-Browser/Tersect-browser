%%
%% Copyright 2022 OXFORD UNIVERSITY PRESS
%%
%% This file is part of the 'oup-authoring-template Bundle'.
%% ---------------------------------------------
%%
%% It may be distributed under the conditions of the LaTeX Project Public
%% License, either version 1.2 of this license or (at your option) any
%% later version.  The latest version of this license is in
%%    http://www.latex-project.org/lppl.txt
%% and version 1.2 or later is part of all distributions of LaTeX
%% version 1999/12/01 or later.
%%
%% The list of all files belonging to the 'oup-authoring-template Bundle' is
%% given in the file `manifest.txt'.
%%
%% Template article for OXFORD UNIVERSITY PRESS's document class `oup-authoring-template'
%% with bibliographic references
%%

%%%CONTEMPORARY%%%
\documentclass[unnumsec,webpdf,contemporary,large]{oup-authoring-template}%
%\documentclass[unnumsec,webpdf,contemporary,large,namedate]{oup-authoring-template}% uncomment this line for author year citations and comment the above
%\documentclass[unnumsec,webpdf,contemporary,medium]{oup-authoring-template}
%\documentclass[unnumsec,webpdf,contemporary,small]{oup-authoring-template}

%%%MODERN%%%
%\documentclass[unnumsec,webpdf,modern,large]{oup-authoring-template}
%\documentclass[unnumsec,webpdf,modern,large,namedate]{oup-authoring-template}% uncomment this line for author year citations and comment the above
%\documentclass[unnumsec,webpdf,modern,medium]{oup-authoring-template}
%\documentclass[unnumsec,webpdf,modern,small]{oup-authoring-template}

%%%TRADITIONAL%%%
%\documentclass[unnumsec,webpdf,traditional,large]{oup-authoring-template}
%\documentclass[unnumsec,webpdf,traditional,large,namedate]{oup-authoring-template}% uncomment this line for author year citations and comment the above
%\documentclass[unnumsec,namedate,webpdf,traditional,medium]{oup-authoring-template}
%\documentclass[namedate,webpdf,traditional,small]{oup-authoring-template}

%\onecolumn % for one column layouts

%\usepackage{showframe}
\usepackage{anyfontsize}
\usepackage{multirow}
\graphicspath{{doc/Images/}} % will allow Latex to find the images folder
\usepackage{graphicx}
\sloppy % allows line spacing stretch to justify the text
% line numbers
%\usepackage[mathlines, switch]{lineno}
%\usepackage[right]{lineno}


\theoremstyle{thmstyleone}%
\newtheorem{theorem}{Theorem}%  meant for continuous numbers
%%\newtheorem{theorem}{Theorem}[section]% meant for sectionwise numbers
%% optional argument [theorem] produces theorem numbering sequence instead of independent numbers for Proposition
\newtheorem{proposition}[theorem]{Proposition}%
%%\newtheorem{proposition}{Proposition}% to get separate numbers for theorem and proposition etc.
\theoremstyle{thmstyletwo}%
\newtheorem{example}{Example}%
\newtheorem{remark}{Remark}%
\theoremstyle{thmstylethree}%
\newtheorem{definition}{Definition}

\begin{document}


\journaltitle{Journal Title Here}
\DOI{DOI HERE}
\copyrightyear{2022}
\pubyear{2019}
\access{Advance Access Publication Date: Day Month Year}
\appnotes{Paper}

\firstpage{1}

%\subtitle{Subject Section}

\title[Short Article Title]{TersectBrowser+ : A single browser for in-depth introgression analysis of resequenced genome datsaets}

\author[1,$\ast$]{Tomasz Kurowski}
\author[2]{Gregory Lupton}
\author[3]{Tanya Stead}
\author[3]{David Oluwasusi}
\author[4]{Gabrielle Baumberg\ORCID{0000-0001-8679-2456}}

\authormark{Author Name et al.}

\address[1]{\orgdiv{Centre for Soil, Agrifood and Biosciences}, \orgname{Cranfield University}, \orgaddress{\street{College Road}, \postcode{MK43 0AL}, \state{Cranfield}, \country{UK}}}
\address[2]{\orgdiv{Department}, \orgname{Organization}, \orgaddress{\street{Street}, \postcode{Postcode}, \state{State}, \country{Country}}}
\address[3]{\orgdiv{Department}, \orgname{Organization}, \orgaddress{\street{Street}, \postcode{Postcode}, \state{State}, \country{Country}}}
\address[4]{\orgdiv{Department}, \orgname{Organization}, \orgaddress{\street{Street}, \postcode{Postcode}, \state{State}, \country{Country}}}

\corresp[$\ast$]{Corresponding author. \href{email:email-id.com}{email-id.com}}

\received{Date}{0}{Year}
\revised{Date}{0}{Year}
\accepted{Date}{0}{Year}

%\editor{Associate Editor: Name}

%\abstract{
%\textbf{Motivation:} .\\
%\textbf{Results:} .\\
%\textbf{Availability:} .\\
%\textbf{Contact:} \href{name@email.com}{name@email.com}\\
%\textbf{Supplementary information:} Supplementary data are available at \textit{Journal Name}
%online.}

\abstract{The field of cultivar genomics has evolved rapidly, with growing emphasis on mapping introgressions—segments of DNA from wild plant species that have been incorporated into cultivated lines. A variety of bioinformatic tools aim to interpret this expanding complexity of plant genomic data, predict introgressions, and identify commercially relevant regions for future breeding. Demand is increasing for tailored tools to interactively explore these complex genomic queries in real time. Created in 2022, the Web-based Tersect Browser can dynamically display introgression patterns by applying flexible set theoretical expressions to sets of sequence variant data. Our 2025 development of Tersect Browser offers multiple front-end extension capabilities so that the user can view and analyze data without leaving the page. We identify previously published introgressions with our tool, illustrating its potential for future research, and enriching plant breeders' understanding of crop cultivar genomics.}
\keywords{introgression, browser, resequenced, SNP}

% \boxedtext{
% \begin{itemize}
% \item Key boxed text here.
% \item Key boxed text here.
% \item Key boxed text here.
% \end{itemize}}

\maketitle
\section{Introduction}

\section{Expanding landscape of plant genomic studies}\label{sec2}

The field of tomato genomics has continued to evolve rapidly, with a growing emphasis on mapping introgressions — segments of DNA from wild tomato species that have been incorporated into cultivated lines. Foundational work such as that by Aflitos et al. (2014) , which sequenced 84 tomato accessions and wild relatives, exposed severe genetic bottlenecks in cultivated tomatoes and laid the groundwork for modern introgression analysis. 

Recent studies have expanded on this base with increasingly sophisticated tools and datasets. Notably, Liu et al. (2023) used tomato pan-genomes built from hundreds of accessions to uncover regions of structural variation and introgression hotspots, many of which affect key traits like fruit size, shape, disease resistance, and flavour. The construction of these pan-genomes enables researchers to observe variation absent in the reference genome, improving trait mapping and introgression detection. 

\subsection{Background on introgression studies}\label{subsec1}

The role of introgression in water-tolerance Kubond et al. (2022), shows how such analysis will be important in identifying variant resistance to climate change.  

To manage and interpret the expanding complexity of genomic data, a wide range of bioinformatic tools have been deployed. Established platforms such as SnpEff, VEP, BEDTools, BEDOPS, CrossMap, and the UCSC Genome Browser continue to support annotation, variant effect prediction, and coordinate translation. Visualization utilities such as the Introgression Browser (Aflitos et al., 2015) and the Tersect Browser (Kurowski and Mohareb, 2019) offer tailored environments for exploring introgression patterns, with Tersect’s set-theoretic framework enabling high-speed, complex genomic queries. 

Meanwhile, long-read sequencing technologies (e.g., PacBio, Hi-C, and optical mapping) as applied in SL4.0 (Hosmani et al., 2019), and innovations in in situ hybridization (Shearer et al., 2014), have markedly improved the resolution of reference genomes—critical for detecting subtle and ancient introgressions. 

Beyond tomato, research in soybean introgression offers valuable cross-crop insights. Studies have shown how introgressed genomic regions from wild Glycine soja into Glycine max improve traits like drought resistance, seed oil composition, and disease resilience. Tools such as SoyFGB v2.0 \footnote{\url{https://sfgb.rmbreeding.cn/index}}, SoyBase\footnote{\url{https://www.soybase.org/}}, and SoyKB\footnote{\url{http://soykb.org/}} provide functional genomics platforms that may serve as models for tomato-specific resources. Graph genome approaches and machine learning models are emerging as promising directions for future introgression prediction and trait association.

Together, these advances paint a dynamic picture of tomato genomics: one where introgression is not only a window into domestication and evolutionary history but also a powerful mechanism for ongoing improvement, including against threats such as climate change. The fusion of high-resolution genomic data, computational methods, and specialized tools enables researchers to explore tomato’s genetic landscape with increasing clarity and sophistication.  
(Introgression hybridisation is an important factor in crop improvement. Biologists need to determine introgressions and identify donor species through visualising genetic distance and phylogenetic relationships based on the whole genome variant data. However, existing software is not suitable for full exploitation of the large publicly available data sets. Tersect Browser is a Web Application that is optimised for generating such visualisations.)

\section{This is an example for first level head}\label{sec3}

\subsection{This is an example for second level head - subsection head}\label{subsec2}

\subsubsection{This is an example for third level head - subsubsection head}\label{subsubsec2}

\paragraph{This is an example for fourth level head - paragraph head}

\section{Conclusion}

Some Conclusions here.

%%%%%%%%%%%%%%

\begin{appendices}

\section{Section title of first appendix}\label{sec11}

\subsection{Subsection title of first appendix}\label{subsec4}

\section{Example of another appendix section}\label{sec13}%

\end{appendices}

\section{Code availability}
The technical documentation is available at XX. The source code for the project is freely accessible on GitHub at XX.

\section{Competing interests}

\section{Author contributions statement}

\section{Acknowledgments}

\begin{thebibliography}{10}
\bibitem{bahdanau2014neural}
Dzmitry Bahdanau, Kyunghyun Cho, and Yoshua Bengio.
\newblock Neural machine translation by jointly learning to align and
  translate.
\newblock {\em arXiv preprint arXiv:1409.0473}, 2014.
\end{thebibliography}

\end{document}
